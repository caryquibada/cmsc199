\section{Introduction}

It has been estimated that 50 million dengue cases infect people per year and there are approximately 2.5 billion people living in dengue-prone areas (tropics and urban settings). The number of dengue cases have been higher than ever. In 1955-1959 only 908 cases of dengue were recorded. A very steep rise in just a span of 60 years. This is mostly caused by the rise in population and having more dense urban living space.\\

The main vector for the dengue virus, mosquito \textit{Aedes Aegypti}, has adapted to the conditions of urban areas in humid and temperate countries. In locations where clear water is abundant and its temperature is just right, the female Aedes lays her eggs. Studying the vectors in this certain environment can be helpful in the planning and mapping of the locations of prime breeding locations and "hotspots" for female Aedes. Having this ability means being able to control Aedes population.\\

Models which utilize spatial aspects and also include time in their construction makes control more efficient. \textbf{MOMA} (Model Of Mosquito Aedes) is a model developed with space and time in mind and also utilizes the geographical outline and formation of a certain neighbourhood. This model uses Geographical Information Systems and Agent-Based Models for mapping the objects in this environment and the latter for giving the Aedes mosquito detailed behaviour

\section{Background}



\subsection{Previous Models}

\textbf{MOMA} is a behavioural model that surveys a large area representing a neighbourhood. It takes into account breeding sites, human density, topology and This sets it apart from previous models.

\subsubsection{Skeeter-Buster} 

Skeeter-Buster's main focus is on breeding site dynamics. It is a stochastic, spatially-explicit model that models cohorts of mosquitoes at a very fine spatial scale, down to the level of individual breeding sites for immature cohorts, or individual houses for adults. Skeeter Buster additionally includes a detailed genetic component, and can therefore model the genetics of Ae. aegypti populations, making it a crucial tool in the evaluation and development of genetic control strategies. 

The main difference of it from MOMA is that Skeeter-Buster does not incorporate blood stocks from humans. It also differs from MOMA because of its scale. Skeeter-Buster focuses on the contents of individual houses and the individual water-containers where the Aedes lays their egg and where their eggs develop into other vectors. 

\subsubsection{SimPopMosq}

Lorem ipsum dolor sit amet, consectetur adipiscing elit. Nam suscipit mauris id imperdiet fringilla. Vivamus ut pharetra enim. Curabitur sed elit erat. Fusce vitae tincidunt purus. Nulla et finibus orci. Etiam faucibus nec nulla vel facilisis. Pellentesque non mauris sed quam blandit dignissim. Maecenas nulla urna, finibus ut justo convallis, mollis tempus diam. Etiam lacinia gravida risus, et suscipit ligula rutrum ut. In ut dictum felis. Etiam ac egestas turpis. Proin enim ante, pharetra a ultrices sit amet, congue sit amet est.

\subsubsection{Inline (In-text) Equations}
Lorem ipsum dolor sit amet, consectetur adipiscing elit. Nam suscipit mauris id imperdiet fringilla. Vivamus ut pharetra enim. Curabitur sed elit erat. Fusce vitae tincidunt purus. Nulla et finibus orci. Etiam faucibus nec nulla vel facilisis. Pellentesque non mauris sed quam blandit dignissim. Maecenas nulla urna, finibus ut justo convallis, mollis tempus diam. Etiam lacinia gravida risus, et suscipit ligula rutrum ut. In ut dictum felis. Etiam ac egestas turpis. Proin enim ante, pharetra a ultrices sit amet, congue sit amet est.

\subsubsection{Display Equations}
Lorem ipsum dolor sit amet, consectetur adipiscing elit. Nam suscipit mauris id imperdiet fringilla. Vivamus ut pharetra enim. Curabitur sed elit erat. Fusce vitae tincidunt purus. Nulla et finibus orci. Etiam faucibus nec nulla vel facilisis. Pellentesque non mauris sed quam blandit dignissim. Maecenas nulla urna, finibus ut justo convallis, mollis tempus diam. Etiam lacinia gravida risus, et suscipit ligula rutrum ut. In ut dictum felis. Etiam ac egestas turpis. Proin enim ante, pharetra a ultrices sit amet, congue sit amet est.


\subsection{Citations}
Lorem ipsum dolor sit amet, consectetur adipiscing elit. Nam suscipit mauris id imperdiet fringilla. Vivamus ut pharetra en


\end{document}  % This is where a 'short' article might terminate



\appendix
%Appendix A
\section{Headings in Appendices}
The rules about hierarchical headings discussed above for
the body of the article are different in the appendices.
In the \textbf{appendix} environment, the command
\textbf{section} is used to
indicate the start of each Appendix, with alphabetic order
designation (i.e., the first is A, the second B, etc.) and
a title (if you include one).  So, if you need
hierarchical structure
\textit{within} an Appendix, start with \textbf{subsection} as the
highest level. Here is an outline of the body of this
document in Appendix-appropriate form:
\subsection{Introduction}
\subsection{The Body of the Paper}
\subsubsection{Type Changes and  Special Characters}
\subsubsection{Math Equations}
\paragraph{Inline (In-text) Equations}
\paragraph{Display Equations}
\subsubsection{Citations}
\subsubsection{Tables}
\subsubsection{Figures}
\subsubsection{Theorem-like Constructs}
\subsubsection*{A Caveat for the \TeX\ Expert}
\subsection{Conclusions}
\subsection{References}
Generated by bibtex from your \texttt{.bib} file.  Run latex,
then bibtex, then latex twice (to resolve references)
to create the \texttt{.bbl} file.  Insert that \texttt{.bbl}
file into the \texttt{.tex} source file and comment out
the command \texttt{{\char'134}thebibliography}.
% This next section command marks the start of
% Appendix B, and does not continue the present hierarchy
\section{More Help for the Hardy}

Of course, reading the source code is always useful.  The file
\path{acmart.pdf} contains both the user guide and the commented
code.

\begin{acks}
  Acknowledgments here...

\end{acks}
